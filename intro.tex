Carrier Sense Multiple Access with Collision Avoidance (CSMA/CA) is the protocol used in wireless local area networks (WLANs) to coordinate transmissions.

Nodes should avoid simultaneous transmissions given that because the medium is shared, these attempts will result in undeciphrable messages to the receivers. This event is referred to as a \emph{collision}. CSMA/CA manages to coordinate transmissions by dividing time in slots, namely; \emph{empty}, \emph{successful} and \emph{colllision} slots, where successful and collision slots contain succesful transmissions or collision events. While the remaining are just tiny empty slots of a fixed time length.

CSMA/CA forces contenders to count down from a randomly generated number (from now on referred to as backoff counter), decrementing it by one per every passing empty slot. When the backoff expires (reaches zero), contenders will attempt transmission. Nevertheless, because the backoff counter is generated at random, there might be cases where two o more contenders simultaneuously attempt transmission and a collision occurs, significantly degrading the throughput of the system as more nodes join the contend for the medium.

The focus of this paper is to describe how it is possible to obtain greater levels of throughput than the achieved by CSMA/CA under optimal parameter configuration, by means of picking a deterministic backoff counter after successful transmissions. This approach is called Carrier Sense Multiple Access with Enhanced Collision Avoidance (CSMA/ECA)~\cite{CSMA_ECA}. Results show that by making this simple modification on the behavior of CSMA/CA, CSMA/ECA preserves the system fairness by equally distributing the system throughput among all contenders. Furthermore, CSMA/ECA is resilient to syncronization flaws on the wireless network cards that can cause a misscount of passing slots (slot drift), as opposed to other MAC protocols~\cite{slotDrift}.
