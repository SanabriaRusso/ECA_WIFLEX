Carrier Sense Multiple Access with Collision Avoidance (CSMA/CA) is the protocol used in wireless local area networks (WLANs) to coordinate transmissions.

Nodes should avoid simultaneous transmissions given that because the medium is shared, these attempts will result in undeciphrable messages to the receivers. This event is referred to as a \emph{collision}. CSMA/CA manages to coordinate transmissions by dividing time in slots, namely; \emph{empty}, \emph{successful} and \emph{colllision} slots, where successful and collision slots contain succesful transmissions or collision events. While the remaining are just tiny empty slots of a fixed time length.

CSMA/CA forces contenders to count down from a randomly generated number (from now on referred to as backoff counter), decrementing it on one per every passing empty slot. When the backoff expires (reaches zero), contenders will attempt transmission. Nevertheless, because the backoff counter is generated at random, there might be cases where two o more contenders simultaneuously attempt transmission and a collision occurs, significantly degrading the throughput of the system as more nodes join the contend for the medium.