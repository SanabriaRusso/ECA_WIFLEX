Collisions are the main cause of throughput degradation in WLANs, specifically because they lead to delays related to retransmission attempts. Therefore, receivers are forced to respond with an \emph{acknowledgement} (ACK) if they are able to effectively decode the message from the sender, which the sender interprets as a indicator for continuing transmissions of the remaining packets in the queue. 

%ACKs are the only feedback CSMA/CA has in order to determine the result of the transmission attempt.

CSMA/CA uses a Binary Exponential Backoff (BEB) algorithm to treat collitions. This algorithm reduces the probability that colliding contenders pick the same backoff counter; thus reducing the collision probability. It is based on the feedback provided by the receivers (ACKs); if none is received, then the range of available backoff counter values doubles. Equation~(\ref{CW}) and~(\ref{BEB}) show how this range increases by augmenting the \emph{backoff stage} ($k$) upon every collision up to the maximum backoff stage ($m=k=5$).

%thus reducing the probability that the colliding contenders randomly pick the same backoff counter.

\begin{subequations}
\begin{align}
	CW(k)= 2^{k}CW_{\min},\label{CW}\\
	B\in[0,CW(k)-1]\label{BEB}
\end{align}
\end{subequations}

In (\ref{CW}), $CW_{\min}$ is called the \emph{minimum contention window} and has a typical value equal to $16$; $k\in[0,m]$ is the backoff stage. This means that $CW(k)$ doubles upon each collision until $k=m$. Therefore, $B$ in~(\ref{BEB}) can be randomly chosen from a wider range of values.
